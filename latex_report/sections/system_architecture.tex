The Smart Proxy system employs a modular architecture with several key components working together to provide comprehensive web security.

\section{Core Components}

\subsection{Proxy Server (main.py)}
The foundation of the system is built on mitmproxy, an advanced HTTPS proxy that enables inspection and modification of web traffic. Our custom addon implemented in main.py intercepts HTTP/HTTPS traffic, extracts relevant content for analysis, and enforces security policies based on detection results.

Key functions of the proxy server include:
\begin{itemize}
    \item Interception of web requests and responses
    \item URL and HTML content extraction
    \item Integration of detection modules
    \item Generation and serving of block pages
    \item Management of bypass tokens and trusted tabs
    \item Whitelist and blacklist enforcement
\end{itemize}

\subsection{ML Phishing Detector (ml\_detector.py)}
This component implements machine learning-based phishing detection using an XGBoost classifier. The model analyzes URL structure, domain characteristics, and HTML content features to identify potential phishing attempts with a confidence score.

\subsection{Scanner Analyzer (scanner\_analyzer.py)}
Complementing the ML approach, this module performs traditional pattern-based analysis by searching for:
\begin{itemize}
    \item Suspicious JavaScript patterns (obfuscation, eval usage)
    \item Hidden iframes with external sources
    \item Known phishing keywords and phrases
    \item Potentially malicious form submission targets
    \item Cross-site scripting (XSS) attempts
\end{itemize}

\subsection{Web Interface (scanner\_app.py)}
A Flask-based web application providing a user-friendly interface for:
\begin{itemize}
    \item Manual URL scanning without visiting websites
    \item Visualization of detection results and risk factors
    \item Whitelist management with full CRUD operations
    \item Configuration of security parameters
\end{itemize}

\section{Data Flow}

The system data flow follows these general steps:

\begin{enumerate}
    \item User browser sends HTTP/HTTPS request via the proxy
    \item Proxy extracts URL and determines if scanning is needed
    \item For HTML responses, content is passed to detection modules
    \item ML model and traditional analyzer evaluate content
    \item If threats are detected, a block page is generated
    \item User can proceed (via bypass) or return to safety
    \item For trusted domains or bypassed sites, content passes through unmodified
\end{enumerate}

\section{Component Interaction Diagram}

\begin{verbatim}
+---------------+        +-----------------+
| User Browser  |<------>| mitmproxy       |
+---------------+        | (main.py)       |
       ^                 +-----------------+
       |                        | |
       |                        v v
       |                 +-----------------+
       |                 | ML Detector     |<------+
       v                 | Scanner Analyzer |      |
+---------------+        +-----------------+      |
| Flask Web App |                                 |
| (scanner_app) |<--------------------------------+
+---------------+
\end{verbatim}

\section{Configuration Management}

The system uses JSON-based configuration files to manage:

\begin{itemize}
    \item ML model paths and confidence thresholds
    \item Whitelist and blacklist entries
    \item Scanning exclusions and performance settings
    \item Block page templates and bypass settings
\end{itemize}

This modular architecture allows for easy extension and customization while maintaining a clear separation of concerns between the proxy functionality, detection algorithms, and user interface components.
